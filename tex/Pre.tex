\documentclass[a4paper, 
	11pt, 
	openany % chapters start on any page. For thesis prints use "twosided"
	]{memoir}

% holds document settings such as title etc
% prototype variable definition:
% \newcommand{\newCommandName}{text to insert\xspace}

\newcommand{\docTitle}{Crimg imvestimgatiom\xspace}
\newcommand{\docAuthor}{QuamtumNoomdle\xspace}
\newcommand{\docOrganization}{Femderal Bumraeu of Crimg\xspace}
\newcommand{\docDepartment}{Weeamboo deparmtment\xspace}
\newcommand{\docDate}{\today\xspace}



%------------------------------------------------- package imports
\usepackage[a4paper, 
			inner = 3.0cm, outer = 3.5cm, 
			top = 2.5cm, bottom = 3.0cm]
			{geometry}  	  					 % page geometry
\usepackage[utf8]{inputenc}   					 % special symbols like ä
\usepackage{dutchcal}          					 % nice Hilbertspace symbols
\usepackage[dvipsnames]{xcolor}   				 % nice color definitions
\usepackage{mdframed}							 % to put frames around environments
\usepackage[english]{babel}    					 % english language package
\usepackage{amsmath}          					 % mathematical symbols and stuff
\usepackage{amssymb} 		  					 % symbols e.g. natural numbers
\usepackage{amsthm}			  					 % math theorem envs
\usepackage{mathtools}        					 % extension of math environments and commands
\usepackage{braket}           					 % for braket notation
\usepackage{graphicx}         					 % include graphics
\usepackage{float}			  					 % forcing position of graphics (e.g. [H])
\usepackage{wrapfig}         					 % figures in text float
\usepackage{enumitem}         					 % nice enumerate environments
\usepackage{nicefrac}         					 % nice tilted fracs
\usepackage{lmodern}          					 % cool font
\usepackage{algpseudocode, algorithm}			 % for pseudocode algorithms
\usepackage{etoolbox}         					 % reformat sections etc.
\usepackage{xspace}								 % escape curly braces in simple commands
\usepackage[unicode = true,	bookmarks = true, 
			bookmarksnumbered = false, 
			bookmarksopen = false, 
			breaklinks = false, 
			pdfborder={0 0 0}, 
			backref = false, colorlinks, 
			linkcolor = red, 
			citecolor = blue]
			{hyperref} 		  					 % hyperlinks
\usepackage{bm}       	 	  					 % bold writing
\usepackage[backend=biber]{biblatex}			 % style=alphabetic for math like citation
\usepackage[outputdir=build]{minted}			 % nicely formatted program code
\usepackage{csquotes}							 % to ensure proper quote formatting in citations


%------------------------------------------------- page styleset
\chapterstyle{chappell}                          % memoir header style
%\setsecheadstyle{\chaptitlefont}                % section font
\setbeforesecskip{-6.5ex plus -1ex minus -.2ex}  % section dimensions
\pagestyle{ruled}

\nouppercaseheads
\graphicspath{ {images/} }

% make chapter a page style of its own
\copypagestyle{chapter}{plain} 
\makeevenfoot{chapter}{}{}{\thepage}
\makeoddfoot{chapter}{}{}{\thepage}
\makeevenhead{chapter}{}{}{}
\makeoddhead{chapter}{}{}{}

% get rid of indent at chapter beginning
\makeatletter
\let\@afterindenttrue\@afterindentfalse
\makeatother

% quote environment formatting
\epigraphfontsize{\small\itshape} 
\setlength\epigraphwidth{11cm}
\setlength\epigraphrule{0pt}


%---------------- Colors
\definecolor{colororange}{HTML}{E65100}		
\definecolor{topbargray}{HTML}{4A5459}	
\definecolor{colordgray}{HTML}{795548}			 
\definecolor{colorhgray}{HTML}{212121}			 
\definecolor{colorgreen}{HTML}{009688} 			
\definecolor{colorlgray}{HTML}{FAFAFA}			 
\definecolor{colorblue}{HTML}{0277BB}	
\definecolor{monokaibg}{HTML}{272822}		
\definecolor{solarizedbg}{HTML}{002b36}	
\definecolor{nativebg}{HTML}{202020}	
\definecolor{grayy}{HTML}{7e7e7e}


%---------------- No indentation environment
\newenvironment{zeroindent}
{\par\setlength{\parindent}{0pt}}
{\par}


%------------------------------------------------- environment settings
\addbibresource{bib/bibliography.bib}
\AtBeginEnvironment{equation*}{\vspace*{0.1cm}}
\AtEndEnvironment{equation*}{\vspace*{0.1cm}}
\newcommand{\crefrangeconjunction}{--} 			 % for referencing multiple figures

%------------------------------------------------- Theorem Styles
% proof end box format
\patchcmd{\endproof}
{\popQED}
{\par\popQED}
{}
{}

\newtheoremstyle{mytheorem}% ⟨name⟩
{16pt}% ⟨Space above⟩
{12pt}% ⟨Space below⟩
{\itshape}% ⟨Body font⟩
{}% ⟨Indent amount⟩
{\bfseries}% ⟨Theorem head font⟩
{.}% ⟨Punctuation after theorem head⟩
{.5em}% ⟨Space after theorem head⟩
{}% ⟨Theorem head spec (can be left empty, meaning ‘normal’)⟩
\theoremstyle{mytheorem}
\newtheorem{theo}{Theorem}

\newtheoremstyle{kor}% ⟨name⟩
{16pt}% ⟨Space above⟩
{12pt}% ⟨Space below⟩
{\itshape}% ⟨Body font⟩
{}% ⟨Indent amount⟩
{\bfseries}% ⟨Theorem head font⟩
{.}% ⟨Punctuation after theorem head⟩
{.5em}% ⟨Space after theorem head⟩
{}% ⟨Theorem head spec (can be left empty, meaning ‘normal’)⟩
\theoremstyle{kor}
\newtheorem{kor}{Corollary}

\newtheoremstyle{mydef}% ⟨name⟩
{16pt}% ⟨Space above⟩
{16pt}% ⟨Space below⟩
{\itshape}% ⟨Body font⟩
{}% ⟨Indent amount⟩
{\bfseries}% ⟨Theorem head font⟩
{.}% ⟨Punctuation after theorem head⟩
{.5em}% ⟨Space after theorem head⟩
{}% ⟨Theorem head spec (can be left empty, meaning ‘normal’)⟩
\theoremstyle{mydef}
\newtheorem{dfn}{Definition}

\newtheoremstyle{myremark}% ⟨name⟩
{12pt}% ⟨Space above⟩
{16pt}% ⟨Space below⟩
{\itshape}% ⟨Body font⟩
{}% ⟨Indent amount⟩
{\bfseries}% ⟨Theorem head font⟩
{}% ⟨Punctuation after theorem head⟩
{.5em}% ⟨Space after theorem head⟩
{}% ⟨Theorem head spec (can be left empty, meaning ‘normal’)⟩
\theoremstyle{myremark}
\newtheorem{rmk}{Remark}
\newtheorem*{rmk*}{Remark}
\newtheorem*{qst*}{Question}

\newtheoremstyle{myex}% ⟨name⟩
{10pt}% ⟨Space above⟩
{16pt}% ⟨Space below⟩
{\itshape}% ⟨Body font⟩
{}% ⟨Indent amount⟩
{\bfseries}% ⟨Theorem head font⟩
{}% ⟨Punctuation after theorem head⟩
{.5em}% ⟨Space after theorem head⟩
{}% ⟨Theorem head spec (can be left empty, meaning ‘normal’)⟩
\theoremstyle{myex}
\newtheorem{ex}{Example}
\newtheorem*{ex*}{Example}


%------------------------------------------------- Math Commands
% Math fonts
\DeclareMathAlphabet{\mathpzc}{OT1}{pzc}{m}{it}  
\DeclareMathAlphabet{\pazocal}{OMS}{zplm}{m}{n}

\newcommand{\el}{ {\scriptstyle{\in}}\; }
\newcommand{\sumel}{ {\scriptscriptstyle{\in}} }
\newcommand*{\defeq}{
	\mathrel{
		\vcenter{\baselineskip0.5ex \lineskiplimit0pt
			\hbox{\scriptsize.}\hbox{\scriptsize.}
		}
	}
	=
} % a three line equation sign, what a rhyme.

\DeclareMathOperator{\tr}{Tr} 
\DeclareMathOperator{\ellip}{\pazocal{E}} 

\newcommand{\probmeas}{\mu}
\newcommand{\lebmeas}{\lambda}

\DeclareMathAlphabet{\cmsy}{OMS}{cmsy}{m}{n}
\newcommand{\Lb}{\cmsy{L}}

% function restriction to subset
\newcommand\restr[2]{{% we make the whole thing an ordinary symbol
		\left.\kern-\nulldelimiterspace % automatically resize the bar with \right
		#1 % the function
		\vphantom{\big|} % pretend it's a little taller at normal size
		\right|_{#2} % this is the delimiter
}}


%------------------------------------------------- Pseudocode settings
\algrenewcommand\algorithmicrequire{\textbf{Input:}}
\algrenewcommand\algorithmicensure{\textbf{Output:}}
\newcommand*\Let[2]{\State #1 $\gets$ #2}
\newcommand*{\Scale}[2][4]{\scalebox{#1}{$#2$}}%


%------------------------------------------------- Minted styleset
\mdfdefinestyle{mystyle}{leftmargin = 0pt,
innerleftmargin = -0.2pt,
innerrightmargin = 0,
innertopmargin = -6.2pt,
innerbottommargin = -6pt,
linewidth=1pt,
linecolor=topbargray}

\BeforeBeginEnvironment{minted}{\begin{mdframed}[style=mystyle]}
\AfterEndEnvironment{minted}{\end{mdframed}}

\setminted{
	baselinestretch=1.2,
	fontsize=\footnotesize,
	% nice styles: monokai, solarized-dark, native. Also adjust bgcolor.
	style=solarized-dark,
    bgcolor=solarizedbg
}


%------------------------------------------------- Set some metadata
\title{\docTitle}
\author{\docAuthor}
\date{\today}






























