\chapter*[Setup]{Setup}
\addcontentsline{toc}{chapter}{Setup}
For help in setting up, consider:
\begin{itemize}
	\item Install Python, version $\geq 3.3$. Install the python requirements from $requirements.txt$ 
	\item Install a Latex distribution, preferably 
	\href{https://miktex.org/howto/install-miktex-unx}{MikTex}. If you use 
	a different Latex distribution, don't blame me for installations being 
	a pain in the ass. 
	\item Set MikTex to installing packages on the fly.
	\item In the MikTex console, install latexmk. You might also need perl for
	\href{http://strawberryperl.com/}{Windows} or  
	\href{https://www.perl.org/get.html}{Linux}.
	\item Use \href{https://code.visualstudio.com/}{Visual Studio Code}. 
	That will make it easier for both of us, because you can use the build 
	settings I took the time to figure out. This template has an unusual 
	build structure and at least one peculiar package. Its got a lot of cool stuff, 
	like automatic compilation, formula preview, build recipes and stuff for other 
	languages to offer.
	\item So, in VS Code, download 
	\href{https://marketplace.visualstudio.com/items?itemName=James-Yu.latex-workshop}
	{\LaTeX workshop} extension. Then integrate the content of $buildrecipe\_settings.json$ 
	into VSCode's $settings.json$. 
	This file contains a compile procedure with all necessary flags and build structure.
	\item \href{
		https://superuser.com/questions/816340/minted-cannot-find-pygmentize-in-texstudio-on-windows-7/1382840
		}{Make minted work}.
	Using a separate python package to do linting and stuff is awkward. 
	Make sure the python package pygments and its subscript pygmentize 
	are properly installed and found in the path before any Latex stuff. 
\end{itemize}

